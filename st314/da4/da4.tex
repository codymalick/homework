\documentclass[10pt,letterpaper]{article}

\usepackage[margin=0.75in]{geometry}
\usepackage{tikz}
\usepackage{graphicx}
\usepackage{amsmath}
\graphicspath{{img/}}
\begin{document}

  \title{Stats 314, Data Analysis \#4}
  \author{Cody Malick\\
  \texttt{malickc@oregonstate.edu}}
  \date{\today}
  \maketitle

\section*{Part I}
\subsection*{a}
A total of 256 students were sampled. While 45 were female, 211 were male. 
Based off this data from the graph, there is evidence that less than 25\% of
students identify as female. 

\subsection*{b}
$p_1-p_2=0$\\
$p_1-p_2<.25$\\

\subsection*{c}
I think that it is an accurate measure as it is an engineers only class. For
that reason, we could take it as a sample of the engineer population. 

\subsection*{d}
It is NOT a random sample.\\
It is of sufficient size.\\


\section*{Part II}
\subsection*{a}
\subsection*{b}
\subsection*{c}
\section*{Part III}
\subsection*{a}
\subsection*{b}
\end{document}
