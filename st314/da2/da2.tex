\documentclass[10pt,letterpaper]{article}

\usepackage[margin=0.75in]{geometry}
\usepackage{tikz}
\usepackage{graphicx}
\usepackage{amsmath}
\graphicspath{{images/}}
\begin{document}

  \title{Stats 314, Data Analysis \#2}
  \author{Cody Malick\\
  \texttt{malickc@oregonstate.edu}}
  \date{\today}
  \maketitle

\section*{Part I}
\subsection*{a}
The dataset with the smallest standard deviation is 'iii.' The reason being
all the numbers are the same. The standard deviation is a quantity calculated
to indicated the extent of deviation for the entire dataset. If they're all the
same number, then the standard deviation of the dataset is 0.

\subsection*{b}
The set with the largest standard deviation will be the one with the most
values far away from the mean. In this case, 'i' would have the largest standard
deviation because of the numbers '7' and '11.' Seven and eleven are the only
differences between the two datasets. Because these two values 'deviate' from
mean more than the values in dataset 'ii,' then 'i' has a larger standard
deviation.

\section*{Part II}
Scheme B would be better because even numbers of both barley populations get
an even amount of water from the river. The main issue with scheme B is that
two of type 1 barley could get all the land by the river, encouraging growth.\\

Scheme B would create two groups of even barley groups. One with both types
getting water, and one with both types not getting water. Scheme B, as stated
above, would be susceptible to an uneven distribution of samples with and without
water.

\section*{Part III}
\subsection*{a}
\subsection*{b}
\subsection*{c}
\subsection*{d}
\subsection*{e}
\subsection*{f}

\section*{Part IV}
\subsection*{a}
\subsection*{b}
\subsection*{c}
\subsection*{d}
\subsection*{e}
\subsection*{f}
\subsection*{g}
\subsection*{h}
\subsection*{i}

\end{document}
